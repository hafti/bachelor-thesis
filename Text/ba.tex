%%%%%%%%%%%%%%%%%%%%%%%%%%%%%%%%%%%%%%%%%%%%%%%%%%%%%%%%%%%%%%%%%%%%%%%%
% Uni Duesseldorf
% Lehrstuhl fuer Datenbanken und Informationssysteme
% Vorlage fuer Bachelor-/Masterarbeiten
% Optimiert fuer den Original-Latex-Kompiler LATEX.EXE (LaTeX=>PS=>PDF)
%%%%%%%%%%%%%%%%%%%%%%%%%%%%%%%%%%%%%%%%%%%%%%%%%%%%%%%%%%%%%%%%%%%%%%%%
% Ueberarbeitung für pdflatex (LaTeX=>PDF)
%%%%%%%%%%%%%%%%%%%%%%%%%%%%%%%%%%%%%%%%%%%%%%%%%%%%%%%%%%%%%%%%%%%%%%%%
% Vorlage Changelog:
% 10.09.2015 (Matthias Liebeck): Nummerierung des Inhaltsverzeichnis nun römisch, Beispiel für einen Anhang eingebaut, \raggedbottom hinter sections eingefügt
%%%%%%%%%%%%%%%%%%%%%%%%%%%%%%%%%%%%%%%%%%%%%%%%%%%%%%%%%%%%%%%%%%%%%%%%
%%%% BEGINN EINSTELLUNG FUER DIE ARBEIT. UNBEDINGT ERFORDERLICH! %%%%%%%
%%%%%%%%%%%%%%%%%%%%%%%%%%%%%%%%%%%%%%%%%%%%%%%%%%%%%%%%%%%%%%%%%%%%%%%%
% Geben Sie Ihren Namen hier an:
\newcommand{\bearbeiter}{Renato Vukovic}

% Geben Sie hier den Titel Ihrer Arbeit an:
\newcommand{\titel}{Chatbots with Deep Learning}

% Geben Sie das Datum des Beginns und Ende der Bachelorarbeit ein:
\newcommand{\beginndatum}{01. August 2018}
\newcommand{\abgabedatum}{31.~Oktober~2018}

% Geben Sie die Namen des Erst- und Zweitgutachters an:
\newcommand{\erstgutachter}{Prof. Dr.~Stefan Harmeling}
\newcommand{\zweitgutachter}{Julius Ramakers}

% Falls Sie die Arbeit zweiseitig ausdrucken wollen,
% benutzen Sie die folgende Zeile mit
% \AN fuer zweiseitigen Druck
% \AUS fuer einseitigen Druck
\newcommand{\zweiseitig}{\AUS}

% Falls die Arbeit in englischer Sprache verfasst 
% werden soll, dann benutzen Sie die folgende Zeile mit
% englisch fuer englische Sprache
% deutsch fuer deutsche Sprache
\newcommand{\sprache}{englisch}

% Hier wird eingestellt, ob es sich bei der Arbeit um eine Bachelor- 
% oder Masterarbeit handelt (unpassendes auskommentieren!):
\newcommand{\arbeit}{Bachelorarbeit}
%~ \newcommand{\arbeit}{Masterarbeit}


%%%%%%%%%%%%%%%%%%%%%%%%%%%%%%%%%%%%%%%%%%%%%%%%%%%%%%%%%%%%%%%%%%%%%%%%
%%%% ENDE EINSTELLUNGEN %%%%%%%%%%%%%%%%%%%%%%%%%%%%%%%%%%%%%%%%%%%%%%%%
%%%%%%%%%%%%%%%%%%%%%%%%%%%%%%%%%%%%%%%%%%%%%%%%%%%%%%%%%%%%%%%%%%%%%%%%

% Die folgende Zeile NICHT EDITIEREN oder loeschen
\input{titelmakros}
\pagenumbering{arabic}
\setcounter{page}{1}

%%%%%%%%%%%%%%%%%%%%%%%%%%%%%%%%%%%%%%%%%%%%%%%%%%%%%%%%%%%%%%%%%%%%%%%%
%%%% BEGINN TEXTTEIL %%%%%%%%%%%%%%%%%%%%%%%%%%%%%%%%%%%%%%%%%%%%%%%%%%%
%%%%%%%%%%%%%%%%%%%%%%%%%%%%%%%%%%%%%%%%%%%%%%%%%%%%%%%%%%%%%%%%%%%%%%%%

%%%%%%%%%%%%%%%%%%%%%%%%%%%%%%%%%%%%%%%%%%%%%%%%%%%%%%%%%%%%%%%%%%%%%%%%
% Text entweder direkt hier hinein schreiben oder, im Sinne der
% besseren Uebersichtlich- und Bearbeitbarkeit mittels \input die
% einzelnen Textteile hier einbinden.
%%%%%%%%%%%%%%%%%%%%%%%%%%%%%%%%%%%%%%%%%%%%%%%%%%%%%%%%%%%%%%%%%%%%%%%%

\section{Introduction}\raggedbottom 

QANet is a state of the art neural network for solving the question answering problem with a single which does not use any recurrent neural nets(RNN) as most of the other state of the art approaches. As RNNs take a lot of time to train this is a advantage in performance. I am trying to use this new architecture in a chatbot where state of the art models use RNNs to see what performance improvement is reached by this.

\begin{figure}[htb]
\begin{center}
  \includegraphics[width=300pt, angle=0]{bilder/QANetArchitecture}
  \caption{An overview of the QANet architecture}\label{fig_QANet}
\end{center}
\end{figure}


\pagebreak
\section{Weiteres Kapitel}\raggedbottom 
\subsection{Unterkapitel}
\subsection{Unterkapitel}

%%%%%%%%%%%%%%%%%%%%%%%%%%%%%%%%%%%%%%%%%%%%%%%%%%%%%%%%%%%%%%%%%%%%%%%%
%%%% ENDE TEXTTEIL %%%%%%%%%%%%%%%%%%%%%%%%%%%%%%%%%%%%%%%%%%%%%%%%%%%%%
%%%%%%%%%%%%%%%%%%%%%%%%%%%%%%%%%%%%%%%%%%%%%%%%%%%%%%%%%%%%%%%%%%%%%%%%

\clearpage

% Entfernen Sie das Kommentar aus der nachfolgenden Zeile, falls Sie einen Anhang in der Arbeit verwenden wollen. Beachten Sie, dass Sie sich im Verlauf der Arbeit mit \ref{...} (z.B. \ref{anhang:zusatz1}) auf den Anhang beziehen.
%\newpage
\appendix
\section{References}

\subsection*{Zusatzteil 1} \label{anhang:zusatz1}

Dies ist ein Anhang.References here.

\clearpage

\bibliography{references}
\bibliographystyle{alphadin}
%\vspace*{\fill}

\clearpage

\listoffigures

\listoftables

%\pagebreak

%\printindex
\end{document}
